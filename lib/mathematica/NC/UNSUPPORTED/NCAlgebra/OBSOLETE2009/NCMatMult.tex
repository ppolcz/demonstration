\CommandEntry
{NCLDUDecomposition[aMatrix, Options]}
{None.}
{NCLDUDecomposition[$X$] yields the LDU decomposition for a square matrix $X$. 
It returns a list of four elements. The first element is the
lower triangular matrix $L$, the second element is the diagonal matrix $D$,
the third element is the upper triangular matrix $U$, and the fourth is
the permutation matrix $P$ (identity if none is provided).}
{X is a square matrix n by n. The default Options are: 
{\tt \{Permuta\-tion $\rightarrow$ \{\}, 
CheckDecomposition  $\rightarrow$ False\}. }
If permutation matrices are to be given, they should be provided 
as {\tt Permutation  $\rightarrow$ $\{l_1$, $l_2$, $\cdots$, $l_n\}$}, 
where each $l_i$ is a list of integers (see the command 
{\tt NCPermu\-tation\-Matrix[]}). 
If {\tt CheckDecomposition} is set to {\tt True}, the function checks 
if $P X P^T$ is identical to $L D U$. Where $P=P_1 P_2 \cdots P_n$, 
and each $P_i$ is the permutation matrix associated with each $l_i$.\\ \\
Suppose X is given by $X = \{ \{ a,b,0\} ,\{ 0,c,d\} ,\{ a,0,d\} \}$. The 
command $\{lo, di, up, P\} = LDU[X]$ returns:
{\footnotesize
\begin{alignat*}{1} 
lo &= \begin{bmatrix}1&0&0\\0&1&0\\1&-b**inv[c]&1\end{bmatrix}\quad\qquad
di = \begin{bmatrix}a&0&0\\0&c&0\\0& 0&d+b**inv[c]**d\end{bmatrix}\\
up &= \begin{bmatrix}1&inv[a]**b& 0\\0&1&inv[c]**d\\0&0&1\end{bmatrix}\qquad
P = \begin{bmatrix}1&0&0\\0&1&0\\0&0&1\end{bmatrix}
\end{alignat*} } \\
As matrix $X$ is $3\times 3$, one can provide 2 permutation matrices. Let 
those permutations be given by $l_1 = \{3,2,1\}$ and $l_2 = \{ 1,3,2\}$, 
that means:
{\footnotesize
\begin{gather*}
P1 = \begin{bmatrix}0&0&1\\0&1&0\\1&0&0\end{bmatrix}\qquad
P2 = \begin{bmatrix}1&0&0\\0&0&1\\0&1&0\end{bmatrix}
\end{gather*}} \\
The command $\{lo,di,up,P\}=LDU[X,Permutation\rightarrow\{l1,l2\}]$ returns:
{\footnotesize
\begin{alignat*}{1} 
lo &= \begin{bmatrix}1&0&0\\0&1&0\\1&-1&1\end{bmatrix}\quad\qquad
di = \begin{bmatrix}d&0&0\\0&a&0\\0& 0&b+c\end{bmatrix}\\
up &= \begin{bmatrix}1&inv[d]**a& 0\\0&1&inv[a]**b\\0&0&1\end{bmatrix}\qquad
P = \begin{bmatrix}0&0&1\\1&0&0\\0&1&0\end{bmatrix} = P_2\: P_1
\end{alignat*} } \\
Readily it can be check that $P^T\; lo\; di\; up\; P \rightarrow X$:
\begin{equation*}
MatMult[Transpose[P], lo, di, up, P] = \{ \{ a,b,0\} ,\{ 0,c,d\} ,\{ a,0,d\} \}
\end{equation*}
}
{{\tt NCLDUDecomposition } automatically assumes invertible any 
expressions (pivot) it needs to be invertible.} 

\CommandEntry
{NCAllPermutationLDU[aMatrix]}
{None.}
{NCAllPermutationLDU[aMatrix] returns the LDU decomposition of a matrix 
for all possible permutations. The code cycles through  all possible 
permutations and calls {\tt NCLDUDecomposition} for each one.}
{A square matrix.}
{The output is a list of all successful outputs from {\tt NCLDUDecomposition}.
Note that some permutations may lead to a zero pivot in the process of 
doing the LDU decomposition. In that case, the LDU decomposition is not 
well defined, actually in Mathematica one gets a lot of $\infty$ signs, 
but this output will not be included in the list of successful outputs.  }

\CommandEntry
{NCInverse[aSquareMatrix]}
{None.}
{NCInverse[$m$] gives a symbolic inverse of a matrix with noncommutative 
entries.}
{$m$ is an $n\times n$ matrix with noncommutative entries.}
{Usually the elements of the inverse matrix ($m^{-1}$) are huge expressions.
We recommend using  {\tt NCSimplifyRational[NCInver\-se[$m$]]} to improve the 
formula you get. 
In some case, {\tt NCSimplifyRatio\-nal[$m^{-1}  m$]} does not 
provide the identity matrix, even though it does equal the identity matrix.
The formula we use for {\tt NCInverse[]} comes from the LDU decomposition. 
Thus in principle it depends on the order chosen for pivoting even the 
inverse of a matrix is unique.  }

\CommandEntry
{NCPermutationMatrix[aListOfIntegers]}
{None.}
{NCPermutationMatrix[aListOfIntegers] returns the permutation matrix 
associated with the list of integers. It is just the identity matrix with 
its columns re-ordered.}
{The encoding specifies where the 1 occurs in each column. e.g.,
\textit{aListOfintegers} $=\{2,4,3,1\}$ represents the permutation matrix
\[ P = \left[ \begin{array}{cccc} 0 & 0 & 0 & 1 \\
1 & 0 & 0 & 0\\ 0& 0 & 1 & 0 \\ 0 & 1 & 0 & 0
\end{array} \right] \] }
{None.}

\CommandEntry
{NCPermutationCheck[SizeOfMatrix, aListOfPermutations]}
{None.}
{If a {\tt ListOfPermutations} is consistent with the matrix size, 
{\tt SizeOfMatrix}, then the output is {\tt True}. If not, the output is 
{\tt False}.}
{The size of a square matrix (an integer) and a list of permutations.}
{None.}

\CommandEntry
{NCDiagonal[aMatrix]}
{None.}
{Returns the elements of the diagonal of a matrix.}
{None.}
{The code is Flatten[MapIndexed[Part,m]].}

