% !TEX TS-program = lualatex
\documentclass[12pt]{article}

\usepackage{luacode}
\begin{luacode}
function paren_array(k,l,m,n)
  for i = k,m do
    for j = l,n do
      tex.sprint("("..i..","..j..")  ")
    end
    tex.print("")  -- force a line break
  end
end

function subscript_array(ind,k,l,m,n)
  for i=k,m do
    for j=l,n do
      tex.sprint(ind.."_{"..i..j.."}")
      if j<n then
        tex.sprint ( "&" )    -- cell separator (ampersand)
      else
        tex.sprint( "\\\\" )  -- end of row (two backslashes)
      end
    end
  end
end
\end{luacode}

\usepackage[ownhref,ccsgyak,thmhu]{polcz}
\usepackage{amsmath,amssymb,mathtools}
\usepackage{xcolor}
\usepackage{lipsum}

\usepackage[
    hyperfootnotes = false,
    colorlinks = true,
    linkcolor = blue,
    urlcolor  = blue,
    citecolor = blue,
    anchorcolor = blue,
    pagebackref = false
    ]{hyperref}
\usepackage[utf8]{inputenc}

\makeatletter
\renewcommand*\env@matrix[1][*\c@MaxMatrixCols c]{%
  \hskip -\arraycolsep
  \let\@ifnextchar\new@ifnextchar
  \array{#1}}
\makeatother

\providecommand{\pfor}[2]{\left[ \foreach \n in {#1}{\n \middle|} #2 \right]}

\makeatletter
\DeclareDocumentCommand\Features{ s m O{0} O{0} m m }{%
  \global\let\MatrixBody\@empty
  \def\TMPvarname{#2}
  \foreach \entry/\index in {#5/100} {%
    \ifnum \index < 100
        \expandafter\g@addto@macro\expandafter\MatrixBody
        \expandafter{%
            \expandafter\TMPvarname\entry &
        }%
    \else
        \expandafter\g@addto@macro\expandafter\MatrixBody
        \expandafter{%
            \expandafter\TMPvarname\entry
        }%
    \fi
  }%
    \begin{bmatrix}%
        \MatrixBody
    \end{bmatrix}%
}
\makeatother

\begin{document}
    $
    % \pmat{a}{2}{4}
    $

    Loop: \\
    \foreach \x/\index in {4,5,4,2/0} {
        \x - \index
        \ifnum \index > 0
            -- kutya
        \fi
        \\
    }

    $
    \bmatrix
        a & b
    \endbmatrix
    $

    $\Features{a}{1,2,3}{}$

    \begin{equation}
        \begin{bmatrix}
            a & c
        \end{bmatrix}
        \begin{bmatrix}[c|c]
            a & c
        \end{bmatrix}
        ~~~~~~~~~~~~~
        \left[\mqty{\xmat*{a}{2}{2}} ~\middle|~ \mqty{\xmat*{a}{2}{2}} \right]
    \end{equation}

    \newcount\i
    \newcount\j

    \newcount\i
    \newcount\j
    \i = 2
    \loop
    \j = 2
    \advance \i by 1
    {
        \loop
        \advance \j by 1
        (\the\i,\the\j)
        \ifnum \j < 6 \repeat
    }
    \ifnum \i < 5
        \global\matrixtoks = \expandafter{\the\matrixtoks \\ }
        \repeat

    \i=0
    \loop
        % increment dummy counter
        \advance\i by 1

        % \j=0
        % \loop

        %     (\the\i,\the\j)

        %     \ifnum\j<6
        % \repeat

        % repeat the loop provided the counter is within specified bound
        \ifnum\i<5
    \repeat


    \begin{equation}
        \qty<1> \dqty{\frac{a}{b},b}
        \qty(a)[b]{c}
    \end{equation}
    \begin{equation}
        \lbrack \lvert
    \end{equation}

    $\foreach \n in {1,3}{\n,}$

    \langol[label]{Kutyagumi}

    \begin{pczLearnEnvironment}*\alpha\asdas[b]{}|c|<d>(required argument)<re>
    \end{pczLearnEnvironment}

    \begin{pczLearnEnvironment}\alpha{}(required argument)<re>
    \end{pczLearnEnvironment}

    \begin{pcolorbox}{Title (Cayley-Hamilton)}[arguments]
        asdasds
    \pcbsub{Subtitle}[arguments]
    PMAT - szabalyosan, loop-okkal:
    $\bmqty{\xmat*{a}{3}{3}}$
    $\bmqty{\pmat*{a}{3}{3}}$,
    $\bmqty{\pmat*{a}[2]{8}[4]{7}}$
    \end{pcolorbox}

    \begin{pcolorbox}{Title}[arguments]\matlab
        asdasds
    \pcbsub{Subtitle}[arguments]\matlab
    \end{pcolorbox}

    \begin{pcolorbox}[notitle]{}[asd][blue][blue!10][blue!20]
        \lipsum[1]
    \end{pcolorbox}

    \begin{pcbexample}
        \lipsum[2]
    \end{pcbexample}

    \begin{cTheorem}[Cayley-Hamilton][arguments]

    \end{cTheorem}

    \Hiba*{123}
    \color{black}
    \TODO{Kutyagumi}(black)

    \Hiba*[blue]
    Kutyagumi asdasdasdasd asd asdas dasd as asd as
    asdasda sd as asd asd asd asd asd ad asd as

    \color{black}
    Kutyagumi(black)

\angol[Magyarazat]{Szo}
\angolI[Magyarazat]{Szo}

\refangol[label]{Kutyaurulek}

\listoftodosangol

\tcbset{enhanced}
\begin{tcolorbox}[breakable,title=My breakable box]
  \lipsum[1-6]
\end{tcolorbox}



\Normal\n
\section{LuaCode}
\newcommand{\luapmat}[4]{%
  $\directlua{ paren_array(\luatexluaescapestring{#1},\luatexluaescapestring{#2},\luatexluaescapestring{#3},\luatexluaescapestring{#4}) }$
}


\newcommand*{\mycommand}{%
  \directlua{for i = 1, 10 do tex.sprint(i .. "\luatexluaescapestring{\hskip1cm}") end}
}

\mycommand

%% call the first function
\directlua{ paren_array(3,4,6,6) }

\bigskip
%% call the second function from within a LaTeX "array" environment
$
A = \left[ \begin{array}{*{5}{c}}
       \directlua{ subscript_array("a",2,3,5,5 ) }
    \end{array} \right]
$

$
\luapmat{2}{3}{4}{4}
$



\newpage
\begin{align*}
    \hmath[remember as=fx]{f(x)}
    &= \int\limits_{1}^{x} \frac{1}{t^2}~dt = \left[ -\frac{1}{t} \right]_{1}^{x}\\
    &= -\frac{1}{x} + \frac{1}{1}\\
    &=
    \hmath[
        overlay={
            % \draw[blue,very thick,->] (fx.south) to[bend right] ([yshift=2mm]frame.west);
            \draw[bourdon,very thick,->] (fx.south) to[bend right] (frame.160);
        }
    ][white]{1-\frac{1}{x}.}
\end{align*}
text
\begin{align}
\hmath[
    overlay={
        % \draw[blue,very thick,->] (fx.south) to[bend right] ([yshift=2mm]frame.west);
        \draw[bourdon,very thick,->] (fx.south) to[bend right] (frame.140);
    }
]{1-\frac{1}{x}}
\end{align}

\newtcbtheorem[number within=section]{mytheo}{My Theorem}%
{colback=green!5,colframe=green!35!black,fonttitle=\bfseries}{th}

\begin{mytheo}{This is my title}{theoexample}
This is the text of the theorem. The counter is automatically assigned and,
in this example, prefixed with the section number. This theorem is numbered with
\ref{th:theoexample} and is given on page \pageref{th:theoexample}.
\end{mytheo}

\tcbmaketheorem{pcbfel}{Feladat}{
    enhanced,
    breakable,
    colback=blue!10 ,
    boxrule=1pt, arc=4pt, left=0.5\pczoversizelength, right=0.5\pczoversizelength, top=6pt, bottom=6pt,
    boxsep=0pt,
    toptitle=4pt,
    bottomtitle=4pt,
    fonttitle=\bfseries,
    coltitle=black,
    colframe=blue!30!black,
    colbacktitle=blue!20,
    subtitle style={
        boxrule=0.4pt,
        colback=blue!20},
    oversize=-\pczoversizelength,
    % lifted shadow={1mm}{-2mm}{3mm}{0.1mm}{black!50!white},
}{pczexample}{pl}

\begin{pcbfel}{\pcbTitleHelper{red}{asdasda}{asdddd}\matlab}{marker}

\end{pcbfel}

Lásd: \ref{pl:marker}

\DeclareDocumentCommand{\pcbtheoremname}{s g g o t\hun}
{%
    \bfseries%
    \IfBooleanTF{#5}%
    {%
        \arabic{#2}. #3.%
    }%
    {%
        #3 \arabic{#2}.%
    }%
    \normalfont%
    \IfBooleanTF{#1}%
    { %
        #4%
    }%
    { %
        (#4)%
    }%
}%

\DeclareDocumentCommand{\pcbtheoreminfo}{s g t\tt}
{%
    \IfBooleanTF{#1}%
    {%
        \hfill%
        {\small\color{bourdon}%
            \IfBooleanTF{#3}%
            {%
                \tt #2%
            }%
            {%
                \itshape #2%
            }%
        }%
    }%
    {%
        {\small\color{bourdon}%
            \IfBooleanTF{#3}%
            {%
                \tt [#2]%
            }%
            {%
                {-- \itshape #2 --}%
            }%
        }%
    }%
}%

\begin{pcb}[title=Kutyagumi][blue]
    This is my own box with a mandatory title and options.
\end{pcb}

\begin{pcb}[title=Kutyagumi,oversize][bourdon]<bourdon!35!yellow>
    This is my own box with a mandatory title and options.
\end{pcb}

\begin{cProposition}*{Kutyagumi}
    This is my own box with a mandatory title and options.
\end{cProposition}

\begin{cExample}*{Title}[asd][asd]

\end{cExample}

\noindent
\pcbtheoremname{pczexample}{Theorem}[Valamilyen példa] \\
\pcbtheoremname{pczexample}{Theorem}[Valamilyen példa]\hun \\
\pcbtheoremname*{pczexample}{Theorem}[Valamilyen példa] \\
\pcbtheoremname*{pczexample}{Theorem}[Valamilyen példa]\hun \\
%
\pcbtheoremname{pczexample}{Theorem}[Valamilyen példa] \pcbtheoreminfo{interp1} \\
\pcbtheoremname*{pczexample}{Theorem}[Valamilyen példa] \pcbtheoreminfo{interp1}\tt \\
\pcbtheoremname{pczexample}{Theorem}[Valamilyen példa]\hun \pcbtheoreminfo*{interp1} \\
\pcbtheoremname*{pczexample}{Theorem}[Valamilyen példa]\hun \pcbtheoreminfo*{interp1}\tt \\

\global\csdef{kutyagumi123}{\bfseries almamartas}
\csuse{kutyagumi123}

\end{document}